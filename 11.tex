\section{Clase 11}
\subsection{Teorema de equiparación generalizado}
\begin{teor}
Dado un Hamiltoniano $H=H(\t )$ cualquiera:
\begin{equation}
  \ev{\t \pdv{H}{\t }}_T=\k T
\end{equation}
donde $\t $ es alguna variable del espacio de fase $p$ o $q$ \footnote{Podemos pensar $\vec{\t }=(\vec{p},\vec{q})$.}.
	
\end{teor}

Hacemos las siguientes suposiciones
\begin{equation}
  H(\t \to \infty)\to \infty
\end{equation}
ó
\begin{equation}
  \t\in D,\quad |D|<\infty
\end{equation}

\begin{dem}
	\begin{equation}
  \ev{\t \pdv{H}{\t }}=\frac{\int\dd\t \t \pdv{H}{\t }e^{-\b H(\t )}}{\int\dd\t  e^{-\b H(\t )}}
\end{equation}
(Cancelando todas las variables del espacio de fase que no entran en el promedio). Notemos que
\begin{equation}
  \pdv{\t}(e^{-\b H(\t )})=-\b e^{-\b H(\t )}\pdv{H}{\t }=-\frac{1}{\k T}e^{-\b H(\t )}\pdv{H}{\t }
\end{equation}
Luego,
\begin{equation}
   \ev{\t \pdv{H}{\t }}=\frac{-\k T\int\dd\t \t\pdv{\t}\left( e^{-\b H(\t )}\right)}{\int\dd \t e^{-\b H(\t )}}
\end{equation}
Integrando por partes
\begin{equation}
  \ev{\t \pdv{H}{\t }}=\frac{\k T\left(\t \eval{e^{-\b H}}_{-\infty}^\infty +\int\dd \t e^{-\b H}\right)}{\int\dd\t e^{-\b H}}
\end{equation}
Si
\begin{equation}
\eval{  \t e^{-\b H}}_{-\infty}^\infty=0
\end{equation}
entonces
\begin{equation}
  \boxed{\ev{\t \pdv{H}{\t }}=\k T}\qquad \qed
\end{equation}
\end{dem}

Notar que para 
\begin{equation}
  H(\t\to\pm\infty)\to \infty
\end{equation}
se cumple que
\begin{equation}
  \t e^{-\b H}\to 0
\end{equation}
con la condición $\t\to\pm\infty$.

Ahora, cuando $\t\in D$ siendo $D$ un dominio acotado, no ocurre que $\t\to\pm\infty$. La manera de modelar esta situación es considerar que hay \textit{una barrera de potencial en los bordes del dominio D.}. Luego, en este caso $H(\t )\to \infty$ con $\t\in D$.

en este caso la integral en $\t $ tiene como límites el borde del dominio. Entonces, el término de borde es
\begin{equation}
\eval{\t e^{-\b H(\t )}}_{\partial D}=0
\end{equation}

\subsection{Teorema del Virial}

En mecánica clásica, existe un teorema análogo al teorema de equipartición llamado teorema del Virial:
\begin{teor}
\begin{equation}
  \ev{p_i\pdv{H}{p_i}}=\ev{q_i\pdv{H}{q_i}}
\end{equation}
\end{teor}
Notar que el promedio $\ev{}$ no necesariamente es un promedio térmico. Es térmico para sistemas en equilibrio térmico que se les puede asociar una temperatura. En general, $\ev{}$ se llama \textit{promedio del ensamble}.

\begin{dem}
	Consideremos
	\begin{equation}
  \ev{\dv{t}(q_ip_i)}=\ev{p_i\dot{q}_i}+\ev{q_i\dot{p}_i}
\end{equation}
y usando las ecuaciones de Hamilton
\begin{equation}
  \dot{q}_i=\pdv{H}{p_i},\qquad \dot{p}_i=-\pdv{H}{q_i}
\end{equation}
obtenemos
\begin{equation}
  \ev{p_i\pdv{H}{p_i}}-\ev{q_i\pdv{H}{q_i}}=\ev{\dv{t}(q_ip_i)}
\end{equation}
Asumimos \textit{ergodicidad}. Luego
\begin{align}
  \ev{\dv{t}(q_ip_i)}&=\lim_{\t\to\infty}\frac{1}{\t }\int_0^\t \left(\dv{t}(p_iq_i)\right)\dd t\label{11.ergo}\\
  &=\lim_{\t\to\infty}\frac{1}{\t }\left(p_i(\t)q_i(\t)-p_i(0)q_i(0)\right)\\
  &=0,\qquad \text{para }\,  \eval{p_i(\t)q_i(\t)}_{\t\to\infty}<0\label{11.otro}
\end{align}
entonces se demuestra que
\begin{equation}
  \ev{\dv{t}(q_ip_i)}=0
\end{equation}
y luego
\begin{equation}
\boxed{  \ev{p_i\pdv{H}{p_i}}=\ev{q_i\pdv{H}{q_i}}}\qquad \qed
\end{equation}
\end{dem}

La demostración de este teorema asume dos puntos importantes:
\begin{enumerate}
	\item \textbf{Ergodicidad}: Es posible igualar un promedio de ensamble (promedio sobre configuraciones posibles) a un promedio temporal. Eso significa que en un tiempo infinito, el sistema explora todas la configuraciones posibles. (\eqref{11.ergo})
	\item En un tiempo infinito, los valores de $p_i$ y $q_i$ son siempre finitos (por el límite de $\t\to\infty$, considerando el $1/\t $para que el límite exista. De manera contraria seria algo como $\infty/\infty$.) (\eqref{11.otro})
\end{enumerate}

\begin{ej}
	Considere el sistema uni-dimensional con potencial
	\begin{equation}
  V(x)=Cx^n,\quad n\geq 1
\end{equation}
\begin{enumerate}
	\item Encuentre $\a$ tal que $\ev{E_c}=\a\ev{V}$, donde $\ev{E_c}$ y $\ev{V}$ son las energías cinéticas y potenciales promedio.
	\item Encuentre $\ev{V}_T$.
\end{enumerate}
\end{ej}

\begin{sol}
\begin{enumerate}
	\item
	El Hamiltoniano viene dado por
	\begin{equation}
  H=E_c+V
\end{equation}
\begin{equation}
  H(p,x)\implies H=\frac{p^2}{2m}+Cx^n
\end{equation}
Entonces, usando el teorema de virial,
\begin{align}
  \ev{p\pdv{H}{p}}&=\ev{x\pdv{H}{x}}\\
  \ev{p\left(\frac{p}{m}\right)}&=\ev{x\left(Cnx^{n-1}\right)}\\
  \underbrace{\ev{\frac{p^2}{m}}}_{2E_c}&=n\underbrace{\ev{x^nC}}_{V}
\end{align}
Nos queda
\begin{align}
  2\ev{E_c}&=n\ev{V}\\
  \ev{E_c}&=\underbrace{\frac{n}{2}}_{\a }\ev{V}
\end{align}
Luego,
\begin{equation}
\boxed{  \a =\frac{n}{2}}
\end{equation}
\item 
\begin{equation}
  2\ev{E_c}_T=n\ev{V}_T=\k T
\end{equation}
luego,
\begin{equation}
 \boxed{ \ev{V}_T=\frac{\k T}{n}}
\end{equation}
Podemos notar que como el potencial es $Cx^n$ el factor del denominador va como el exponente del potencial.
\end{enumerate}
\end{sol}

\underline{\textbf{Nota:}} Si quisiéramos calcular el promedio de $E_c$,
\begin{align}
  2\ev{E_c}&=n\ev{V}\\
  \ev{E_c}&=\frac{n}{2}\left(\frac{\k T}{n}\right)
\end{align}
\begin{equation}
\boxed{  \ev{E_c}=\frac{1}{2}\k T}
\end{equation}
donde vemos que como $E_c$ es cuadrática en $p$, el resultado está dividido en $2$.



















