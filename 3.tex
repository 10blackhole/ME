\section{Clase 3}\label{clase:3}
\subsection{Principio fundamental de la termodinámica}
\begin{tcolorbox}
La configuración total \textbf{maximiza la entropía}, sujeta a restricciones macroscópicas que dependen de las cantidades fluctuantes conservadas.
\end{tcolorbox}

En la Clase \ref{clase:2} vimos que si suponemos una configuración $N_s$ sistemas los que pueden existir en $m$ estados diferentes, siendo sus ocupaciones el conjunto:
\begin{equation}
  \{n_i\}_{i=1}^m
\end{equation}
la ignorancia $I$, o degenerancia $\Omega$ esta dada por
\begin{equation}
  \boxed{I=\frac{N_s!}{\Pi_{i=1}^mn_i}}
\end{equation}
Se define la entropía $S$ como
\begin{equation}
  \boxed{S=\kappa \ln I}
\end{equation}
Usando Stearling, se tiene
\begin{align}
  \frac{S}{\kappa}=N_s\ln (Ns)-Ns-\left[\sum_{i=1}^m\left(n_i\ln (n_i)-n_i\right)\right]
\end{align}
Usando que $\sum_{i=1}^m=N_s$,
\begin{align}
  \frac{S}{\kappa}&=N_s\ln (Ns)-\summ n_i\ln(n_i)\quad /\frac{1}{N_s}\\
  \frac{S}{\kappa N_s}&=\ln(N_s)-\summ\left(\frac{n_i}{N_s}\right)\ln (n_i)\label{3.1si}
\end{align}
Usando que
\begin{equation}
  \ln(n_i)=\ln\left(\frac{n_i}{N_s}N_s\right)=\ln\left(\frac{n_i}{N_s}\right)+\ln(N_s)
\end{equation}
reemplazando en \eqref{3.1si},
\begin{align}
  \frac{S}{\kappa N_s}&=\ln (N_s)-\summ\left(\frac{n_i}{N_s}\right)\left[\ln(\frac{n_i}{N_s})+\ln (N_s)\right]\\
  &=\ln(N_s)-\summ \left(\frac{n_i}{N_s}\right)\ln(\frac{n_i}{N_s})-\summ \left(\frac{n_i}{N_s}\right)\ln(N_s)
\end{align}
pero usando \eqref{sum Pi1}
\begin{equation}
  \summ \left(\frac{n_i}{N_s}\right)=\summ P_i=1
\end{equation}
así,
\begin{align}
  \frac{S}{\kappa N_s}&=\ln(N_s)-\summ \left(\frac{n_i}{N_s}\right)\ln(\frac{n_i}{N_s})-\ln(N_s)\\
  &=-\summ\left(\frac{n_i}{N_s}\right)\ln(\frac{n_i}{N_s})
\end{align}
\begin{equation}
  \Rightarrow \boxed{\frac{S}{\kappa N_s}=-\summ\left(\frac{n_i}{N_s}\right)\ln(\frac{n_i}{N_s})}
\end{equation}
Definimos la probabilidad de que un sistema esté en el estado $i$ como \eqref{sum Pi1}
\begin{equation}
\boxed{P_i=\frac{n_i}{N_s}}
\end{equation}
y la entropia de Shannon como
\begin{equation}
  \boxed{\frac{S}{\kappa N_s}=-\summ P_i\ln (P_i)}
\end{equation}
Por el principio fundamental del termodinámica, sin imponer conservación de las cantidades conservadas. Pero considerando
\begin{equation}
  \summ P_i=1
\end{equation}
maximizamos la entropía
\begin{align}
  L&=-\summ P_i\ln(P_i)+\lambda\left(\summ P_i-1\right)
\end{align}
donde $\lambda$ son los multiplicadores de Lagrange y $\summ P_i-1$ es una restricción holonómica.
\begin{align}
  \left(\pdv{L}{P_i}\right)_{P_i,\lambda}&=0,\qquad \forall i\neq j\label{3.1}\\
  \left(\pdv{L}{\lambda}\right)_{P_i}&=0\label{3.2}
\end{align}
De \eqref{3.1}
\begin{align}
  -\ln(P_i)-P_i\frac{1}{P_i}+\lambda&=0\\
  \Rightarrow \ln(P_i)&=\lambda-1
\end{align}
obtenemos
\begin{equation}
  P_i=e^{\lambda-1}
\end{equation}
notemos que esta expresión es independiente de $i$.
De \eqref{3.2},
\begin{align}
  \summ P_i-1&=0\\
 \Rightarrow \summ P_i&=1\\
 \Rightarrow \summ P_i&=mP=me^{\lambda-1}=1
\end{align}
luego
\begin{equation}
  P_i=\frac{1}{m}\quad \Rightarrow\quad n_i=\frac{N_s}{m}\quad\Rightarrow\quad \lambda=-\ln(n)+1
\end{equation}
\corr{Sin imponer restricciones de conservación, excepto que la suma de las probabilidades de los estados es igual a $1$, obtenemos que la configuración que maximiza la entropía es la distribución uniforme de estados equiprobables}. La ocupación de cada uno de los $m$ estados es:
\begin{equation}
  \boxed{n_i=\frac{N_s}{m}}
\end{equation}

\bako{Si imponemos conservación de la energía},
\begin{itemize}
	\item Cada estado $i$ tiene energía $E_i$
	\item Si las ocupaciones de dichos estados son $n_i$, la energía total de la configuración es
	\begin{equation}
  E=\summ E_in_i
\end{equation}
\item Agregamos la conservación de $E$ como una restricción al funcional de entropía.
\end{itemize}

Sea $S_E$ la restricción
\begin{equation}
  S_E=\b \k \left(E-\summ E_in_i\right)
\end{equation}
con $\b$ un multiplicador de Lagrange. Consideremos
\begin{equation}
  L_E=\frac{S_E}{\k N_s}=\b \left(\bar{E}-\summ E_iP_i\right)
\end{equation}
donde 
\begin{equation}
  \bar{E}=\frac{E}{N_s}
\end{equation}
es la energía promedio del sistema.

Luego, nuestro Lagrangeano queda
\begin{equation}
  L=-\summ P_i\ln(P_i)-\lambda \left(\summ P_i-1\right)-\b \left(\summ E_iP_i-\bar{E}\right)
\end{equation}
Ahora,
\begin{align}
  \left(\pdv{L}{P_i}\right)_{P_i,\lambda}&=\ln(P_i)-1-\lambda-\b E_i\\
  \left(\pdv{L}{\b}\right)_{P_i,\lambda}&=\summ E_iP_i=\bar{E}\\
  \left(\pdv{L}{\lambda}\right)_{P_i,\b}&=\summ P_i=1
\end{align}
\begin{equation}
  \ln(P_i)=-\lambda-\b E_i-1
\end{equation}
entonces
\begin{equation}
\boxed{  P_i=e^{-\lambda-\b E_i-1}}
\end{equation}
\begin{align}
  P_i&=e^{-\lambda-1}e^{-\b E_i}\\
  \summ P_i&=1\Rightarrow \summ e^{-\lambda-1}e^{-\b E_i}=1\\
  \Rightarrow (e^{-\lambda-1})\summ e^{-\b E_i}=1\\
  \Rightarrow e^{-\lambda-1}&=\frac{1}{\summ e^{-\b E_i}}
\end{align}
tenemos
\begin{equation}
  \boxed{P_i=\frac{e^{-\b E_i}}{\summ e^{-\b E_i}}}\quad \Rightarrow\quad \boxed{Z=\summ e^{-\b E_i}}\quad \Rightarrow\quad \boxed{P_i=\frac{e^{-\b E_i}}{Z}}
\end{equation}
En el caso de imponer conservación de la energía, la configuración solo puede explorar conjuntos de ocupaciones $\{n_i\}_{i=1}^n$ que tengan la misma energía total.

En general, la configuración equiprobable no es alcanzable necesariamente, ya que tiene una energía total particular, que puede ser diferente a la energía inicial dada.

En particular, la energía de la configuración equiprobable es
\begin{align}
  E_{\rm equi}&=\summ E_in_i\\
  &=\summ E_i\frac{N_s}{m}
\end{align}
esto implica
\begin{equation}
  \boxed{E_{\rm equi}=\frac{N_s}{n}\summ E_i}
\end{equation}
Si $E\neq E_{\rm equi}$, la configuración equiprobable es inalcanzable. Para $E$ consevado, la probabilidad de que un sistema esté en el estado $i$ es
\begin{equation}
 \boxed{P_i=\frac{e^{-\b E_i}}{Z}}
\end{equation}
donde
\begin{equation}
\boxed{ Z=\summ e^{-\b E_i}}
\end{equation}
se conoce como la \textbf{función de partición} y $e^{-\b E_i}$ se llama \textbf{factor de Boltzmann} y tiene la interpretación de un peso estadistico.
\begin{itemize}
	\item $Z$ es la suma de todos los pesos
	\item La ocupación del estado $i$ es
	\begin{equation}
\boxed{  n_i=N_sP_i=N_s\frac{e^{-\b E_i}}{Z}}
\end{equation}
\end{itemize}

\subsection{Conservación de energía y carga $Q$}
$Q$ puede ser número de partículas, carga eléctrica, momento angular, magnetización, etc...

$Q$ es una cantidad macroscópica, extensiva, diferente de $(E,V,S)$ que caracteriza la configuración macroscópica y que se conserva entre las distintas configuraciones.

El Lagrangeano se verá
\begin{equation}
  L=-\summ P_i\ln(P_i)-\underbrace{\lambda \left(\summ P_i-1\right)}_{\summ P_i=1}-\underbrace{\b\left(\summ E_iP_i-\bar{E}\right)}_{\text{Conservación de la energía}}-\underbrace{\b \m\left(\summ Q_iP_i-\bar{Q}\right)}_{\text{Conservación de la carga}}
\end{equation}
con $\bar{Q}$ la carga promedio del sistema.

Ahora
\begin{align}
  -\ln(P_i)-1-\lambda-\b E_i-\b\m Q_i&=0\\
  P_i=e^{\lambda-1-\b E_i-\b\m Q_i}&=0\\
  P_i&=e^{-\lambda-1}e^{-\b(E_i+\m Q_i)}
\end{align}
usando $\summ P_i=1$,
\begin{align}
  \summ e^{-\b (E_i+\m Q_i)}=\frac{1}{e^{-\lambda-1}}
\end{align}
Definimos
\begin{equation}
\boxed{  Z_{\rm GC}=\summ e^{-\b (E_i+\m Q_i)}}
\end{equation}
entonces
\begin{equation}
 \boxed{ P_i=\frac{e^{-\b (E_i+\m Q_i)}}{Z_{\rm GC}}}
\end{equation}
donde $Z_{\rm GC}$ es la \textbf{función partición Gran Canónica} y corresponde a la suma de los pesos estadisticos $e^{-\b (E_i+\m Q_i)}$. $n_i=N_sP_i$ es la ocupación de los estados.

\subsection{Ensambles, funciones de partición y potenciales termodinámicos}
\begin{itemize}
	\item Maximizando la entropía, sujeto a diferentes restricciones, hemos encontrado la probabilidad de ocupación de los estados posibles de los sistemas.
	\item Dichas probabilidades pueden interprearse como pesos estadisticos para cada estado, normalizados por una función partición.
	\item Dependiendo de las cantidades conservadas las funciones partición son distintas.
	\item El conjunto de configuraciones posibles dadas las restricciones, definen la noción de ensamble.
	\item A cada tipo de ensamble, especificado por ciertas cantidades conservadas, le corresponde una función partición.
\end{itemize}


\begin{center}
\begin{tabular}{|c|c|c|c|c|}
\hline
  Ensamble & Can. fluctuantes conservadas & Can. fijas & Función partición & Pot. termodinámico  \\
  \hline\hline
  Microcanónico & ----- & $E,V,Q$& $I$& $S$ \\\hline
  Canónico&$E$ &$T,V,Q$&$z=\summ e^{-\b E_i}$&$F$\\\hline
  Gran Canónico&$E,Q$&$T,V,\m$&$z=\summ e^{-\b(E_i+\m Q_i)}$&$\Omega_{\rm GC}$\\\hline
  Isobárico&$E,V$&$T,P,Q$&&\\\hline
\end{tabular}
\end{center}
donde
\begin{equation}
  \b=\frac{1}{T}
\end{equation}

Notas que $\b,\m$ y $P$ son intensivas.

$T=\frac{1}{\b}$ viene dado por el multiplicador de Lagrange asociado a la conservación de energía.

\subsection{Ensamble termodinámico*}
En mecánica estadística primero uno debe considerar que cantidades desea mantener fijas, y que cantidades desea permitir que varíen (pero con el constraint de que su promedio sea algún valor). Esta elección define el \textbf{ensamble}. Los ensambles difieren en las cantidades que varían. Si un cantidad es permitida que varíe, entonces un multiplicador de Lagrange determina la cantidad promedio para cada ensamble.

\begin{center}
\begin{tabular}{|c|c|c|}
\hline
  Ensamble & Energía & Cargas  \\
  \hline
  Microcanónico &Fija& Fija \\\hline
  Canónico&Varía &Fija\\\hline
  Gran Canónico&Varía&Varía\\\hline
\end{tabular}
\end{center}






































