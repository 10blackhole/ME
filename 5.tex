\section{Clase 5}
\subsection{Relaciones termodinámicas}
Hasta el momento hemos obtenido las siguientes relaciones termodinámicas
\begin{tcolorbox}
\begin{itemize}
	\item Relación de Euler
	\begin{equation}
  E=TS-PV+\m Q
\end{equation}
\item Energía libre de Hemholtz \begin{equation}
  F=-\k T\ln(\zc)=E-TS
\end{equation}
\item Potencial gran canónico
\begin{equation}\label{5.star}
  \Ogc=-PV=-\k T\ln (\zgc)=E-TS-\m Q
\end{equation}
\item Energía libre de Gibbs
\begin{equation}
  G=\m Q=E-TS+PV
\end{equation}
\end{itemize}
\end{tcolorbox}

Por otro lado, las funciones de partición de cada ensamble y la relación de $E$ y $Q$ con ellas vienen dadas por\footnote{Notar que no escribimos la barra de promedio en $\bar{E}$ y $\bar{Q}$ ya que se sobreentiende donde corresponde.}
\begin{tcolorbox}
\begin{equation}
  \zc =\summ e^{-\b E_i},\qquad \b =\frac{1}{T}
\end{equation}
\begin{equation}
  \zgc=\summ e^{-\b E_i-\a Q_i},\qquad \a=-\frac{\m }{T}
\end{equation}

\begin{equation}\label{5.box1}
  \bar{E}=\k T^2\left(\pdv{T}\ln(Z)\right)_{\m/T}
\end{equation}

\begin{equation}\label{5.box2}
 \bar{Q}=\k T\left(\pdv{\m }\ln(\zgc)\right)_T
\end{equation}
\end{tcolorbox}

\subsection{Derivación de las leyes de la termodinámica}
De \eqref{5.star}
\begin{equation}
  \Ogc=-PV=-\k T\ln (\zgc)=E-TS-\m Q
\end{equation}
despejando para $S$
\begin{equation}
  S=\k \ln(\zgc(\m /T,T,V))+\frac{E}{T}-\left(\frac{\m }{T}\right)Q
\end{equation}
luego,
\begin{align}\label{5.ds}
  \dd S&=\left(\k\pdv{\ln \zgc}{T}-\frac{E}{T^2}\right)\dd T+\left(\k\pdv{\ln\zgc}{(\m/T)}-Q\right)\dd Q+\k\pdv{\ln\zgc}{V}\dd V+\frac{\dd E}{T}-\frac{\m\dd Q}{T}
\end{align}
pero 
\begin{equation}
  PV= T\ln\zgc \qquad \implies\qquad \ln\zgc=\frac{P}{T}V
\end{equation}
tomando la derivada parcial con respecto a $V$
\begin{equation}
  \pdv{\ln\zgc}{V}=\frac{P}{T}
\end{equation}
reemplazando en \eqref{5.ds},
\begin{align}
  \dd S=\left(\k\pdv{\ln \zgc}{T}-\frac{E}{T^2}\right)\dd T+\left(\k\pdv{\ln\zgc}{(\m/T)}-Q\right)\dd Q+\frac{P}{T}\dd V+\frac{\dd E}{T}-\frac{\m\dd Q}{T}
\end{align}
usando \eqref{5.box1} y \eqref{5.box2} se cancelan los primeros dos términos, obteniendo
\begin{equation}
  \dd S=\frac{P}{T}\dd V+\frac{\dd E}{T}-\frac{\m\dd Q}{T}
\end{equation}
\begin{equation}
 \boxed{ T\dd S=P\dd V+\dd E-\m\dd Q}
\end{equation}
conocida como la \textbf{segunda ley de la termodinámica}.

\subsection{Relaciones de equilibrio termodinámico}
Consideremos dos sistemas $C_1$ y $C_2$, los cuales pueden intercambiar energía y carga, y ademas pueden modificar su volumen. De las leyes de conservación se tiene
\begin{equation}
  \boxed{\dd E_1=-\dd E_2,\qquad \dd Q_1=-\dd Q_2,\qquad \dd V_1=-\dd V_2}
\end{equation}
también se tiene
\begin{equation}
 \boxed{ \dd S_{\rm total}=\dd S_1+\dd S_2}
\end{equation}
Es decir, el cambio en la entropía es la suma de los cambios de la entropía de cada sistema.

Ahora
\begin{equation}
  \boxed{\dd S_i=\frac{P_i}{T_i}\dd V_i+\frac{1}{T_i}\dd E_i-\frac{\m_i}{T_i}\dd Q_i}
\end{equation}
Esto es el cambio de la entropía en cada sistema dado por la segunda ley de la termodinámica. Por lo tanto
\begin{equation}
  \boxed{\dd S_{\rm total}=\dd S_1+\dd S_2=\left(\frac{P_1}{T_1}-\frac{P_2}{T_2}\right)\dd V_1+\left(\frac{1}{T_1}-\frac{1}{T_2}\right)\dd E_1+\left(\frac{\m_1}{T_1}-\frac{\m_2}{T_2}\right)\dd Q_1}
\end{equation}
Pero del principio de la máxima entropía, la configuración maximiza la entropía (sujeto a las restricciones de conservación). Luego, en todo proceso termodinámico, $\dd S_{\rm total}\geq 0$.

La condición de equilibrio termodinámico requiere $\dst=0$. Por lo tanto la condición de equilibrio establece:
\begin{equation}
  \frac{P_1}{T_1}=\frac{P_2}{T_2},\qquad \frac{1}{T_1}=\frac{1}{T_2},\qquad \frac{\m_1}{T_1}=\frac{\m_2}{T_2}
\end{equation}
entonces
\begin{align}
  P_1&=P_2\\
   T_1&=T_2\\
    \m_1&=\m_2
\end{align}
esto implica también que le intercambio de energía interna o calor ($E$) se produce desde el sistema de mayor temperatura hacia el de menor temperatura, el intercambio de cargas ($Q$) ocurre desde el sistema de mayor potencial químico hacia el de menor potencial químico y el cambio de volumen ($V$)  ocurre de modo que el sistema con mayor presión se expande y de menor presión se contrae.

\subsection{Otras relaciones termodinámicas}
Por un lado
\begin{equation}
  E=TS-PV+\m Q
\end{equation}
si derivamos
\begin{equation}
  \dd E=\dd TS+T\dd S-\dd PV-P\dd V+\dd\m Q+\m \dd Q
\end{equation}
además de la segunda ley,
\begin{equation}
  \dd E=T\dd S-P\dd V+\m \dd Q
\end{equation}
reemplazamos $\dd E$ en lo anterior y obtenemos
\begin{equation}
  \boxed{S\dd T-V\dd P+Q\dd\m =0}
\end{equation}

\subsubsection{Acerca de los potenciales termodinámicos}
El potencial gran canónico viene dado por
\begin{equation}
  \Omega=E-\m Q-TS
\end{equation}
derivamos
\begin{equation}
  \dd\Omega=\dd E-\dd\m Q-\m \dd Q-\dd TS-T\dd S
\end{equation}
usando de nuevo la segunda ley
\begin{equation}
  T\dd S=P\dd V+\dd E-\m\dd Q
\end{equation}
y obtenemos
\begin{equation}
 \boxed{ \dd \Omega=-S\dd T-P\dd V-Q\dd\m}
\end{equation}
Notar que el ensamble gran canónico tiene $(T,V,\m )$ dijos y $(E,Q)$ fluctuantes. Luego,
\begin{equation}
  \Omega=\Omega(T,V,\m )
\end{equation}
y
\begin{equation}
  \dd\Omega=\left(\pdv{\Omega}{T}\right)_{V,\m }\dd T+\left(\pdv{\Omega}{V}\right)_{T,\m }\dd V+\left(\pdv{\Omega}{\m }\right)_{T,V }\dd \m 
\end{equation}
Por lo tanto,
\begin{align}
  \left(\pdv{\Omega}{T}\right)_{V,\m }&=-S\\
  \left(\pdv{\Omega}{V}\right)_{T,\m }&=-P\\
  \left(\pdv{\Omega}{\m }\right)_{T,V }&=-Q
\end{align}

Para la energía libre de Hemholtz en su versión integrada viene dada por
\begin{equation}
  F=E-TS
\end{equation}
\begin{equation}
  \dd F=\dd E-\dd TS-T\dd S
\end{equation}
Usando nuevamente la segunda ley, obtenemos
\begin{equation}
  \boxed{\dd F=-P\dd V+\m \dd Q-S\dd T}
\end{equation}
En la definición del ensamble canónico, $(T,V,Q)$ estan fijos, y $E$ fluctua. Luego, $F=F(T,V,Q)$, por lo tanto
\begin{equation}
  \dd F=\left(\pdv{F}{T}\right)_{V,Q}\dd T+\left(\pdv{F}{V}\right)_{T,Q}\dd V+\left(\pdv{F}{Q}\right)_{T,V}\dd Q
\end{equation}
luego,
\begin{align}
  \left(\pdv{F}{T}\right)_{V,Q}&=-S\\
  \left(\pdv{F}{V}\right)_{T,Q}&=\m \\
  \left(\pdv{F}{Q}\right)_{T,V}&=-P
\end{align}

Para la energia libre de Gibbs,
\begin{equation}
  G=E+PV-TS
\end{equation}
\begin{equation}
  \dd G=\dd E+\dd PV+P\dd V-\dd T S-T\dd S
\end{equation}
Haciendo un procedimiento similar al de los casos anteriores, se tiene
\begin{equation}
  \boxed{\dd G=-S\dd T+V\dd P+\m \dd Q}
\end{equation}

en el ensamble isobárico-isotérmico, $(T,P,Q)$ están fijas y $(E,V)$ fluctúa. Luego, $G=G(T,P,Q)$, y por tanto
\begin{equation}
  \dd G=\left(\pdv{G}{T}\right)_{P,Q}\dd T+\left(\pdv{G}{P}\right)_{T,Q}\dd P+\left(\pdv{G}{Q}\right)_{P,T}\dd Q
\end{equation}
por tanto
\begin{align}
  \left(\pdv{G}{T}\right)_{P,Q}&=-S\\
  \left(\pdv{G}{P}\right)_{T,Q}&=V\\
  \left(\pdv{G}{Q}\right)_{P,T}&=\m 
\end{align}

La \textbf{entalpía} se define como
\begin{equation}
  H=E+PV
\end{equation}
entonces 
\begin{equation}
  \dd H=\dd E+\dd PV+P\dd V
\end{equation}
reemplazando $\dd E$ de la segunda ley,
\begin{equation}
  \boxed{\dd H=T\dd S+V\dd P+\m \dd Q}
\end{equation}
Por tanto en el \textbf{ensamble de Joule-Thompson}, del cual la es su potencial termodinámico, tiene $(S,P,Q$ fijos y $V$ fluctuante, es decir, $H=H(S,P,Q)$. Luego,
\begin{equation}
  \dd H=\left(\pdv{H}{S}\right)_{P,Q}\dd S+\left(\pdv{H}{P}\right)_{S,Q}\dd P +\left(\pdv{H}{Q}\right)_{P,S}\dd Q
\end{equation}
Luego, nos queda
\begin{align}
  \left(\pdv{H}{S}\right)_{P,Q}&=T\\
  \left(\pdv{H}{P}\right)_{S,Q}&=V\\
  \left(\pdv{H}{Q}\right)_{P,S}&=\m 
\end{align}

Finalmente, para el ensamble microcanónico, las cantidades fijas son $(E,V,Q)$ y de la segunda ley se tiene
\begin{equation}
  \dd E=T\dd S-P\dd V+\m \dd Q
\end{equation}
Luego, se puede despejar
\begin{equation}
  \dd S=\left(\frac{1}{T}\right)\dd E+\left(\frac{P}{T}\right)\dd V-\left(\frac{\m }{T}\right)\dd Q
\end{equation}
y considerando que a $(E,V,Q)$ fijos, se tiene que $S=S(E,V,Q)$ y 
\begin{equation}
  \dd S=\left(\pdv{S}{E}\right)_{V,Q}\dd E+\left(\pdv{S}{V}\right)_{E,Q}\dd V+\left(\pdv{S}{Q}\right)_{V,E}\dd Q
\end{equation}
se tiene que
\begin{align}
  \left(\pdv{S}{E}\right)_{V,Q}&=\frac{1}{T}\\
  \left(\pdv{S}{V}\right)_{E,Q}&=\frac{P}{T}\\
  \left(\pdv{S}{Q}\right)_{V,E}&=-\frac{\m }{T}
\end{align} 

\subsection{Resumen: ensambles y potenciales termodinámicos}
Como resumen, tenemos la siguiente tabla que relaciona los distintos ensambles con sus potenciales termodinámicos y variables fijas:


\begin{center}
\begin{tabular}{|c|c|c|}
\hline
  Ensamble & Potencial termodinámico & Variabes fijas  \\
  \hline
  Microcanónico &Entropía $(S)$& $(E,V,Q)$ \\\hline
  Canónico&Energía libre de Hemholtz $(F)$ &$(T,V,Q)$\\\hline
  Gran Canónico&Potencial gran canónico $(\Ogc)$&$(T,V,\m )$\\\hline
  Isobárico-Isotérmico&Energía libre de Gibbs $(G)$&$(T,P,Q)$\\\hline
  Joule-Thompson&Entalpía $(H)$&$(S,P,Q)$\\\hline
\end{tabular}
\end{center}











































































