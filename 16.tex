\section{Clase 16}
\subsection{Gas de bosones y fermiones sin masa a temperatura $T$}
Veremos la derivación de la ley de radiación de cuerpo negro de Plank a partir de un ejemplo.

\begin{ej}
	Considere un gas de fotones (bosones sin masa) a temperatura $T$. Encuentre
	\begin{enumerate}
		\item La densidad de energía $E/V$.
		\item La densidad de número de partículas $N/V$.
		\item La densidad de energía por unidad de frecuencia de la radiación, o también llamada densidad espectral de energía (Bv(T) o función del cuerpo negro o función de Plank.)
	\end{enumerate}
\end{ej}

\begin{sol}
	De lo visto en clases tenemos,
\begin{equation}\label{16.1}
  \frac{J}{V}=(2s+1)\int\frac{\dd^3p}{(2\p\hbar)^3}j(p)f(\epsilon(p))
\end{equation}
donde $j(p)$ es una cantidad física para una partícula con magnitud de momentum $p$ y 
\begin{equation}
  f(\epsilon(p))=\frac{e^{-\b (\epsilon(p)-\m )}}{1\mp e^{-\b (\epsilon(p)-\m )}}
\end{equation}
donde $\m$ corresponde al potencial químico, $\epsilon(p)$ es la relación de dispersión y el signo negativo es para bosones y el positivo para fermiones.

Recordemos que la energía promedio $E$ la obtenemos derivando con respecto a $\b$ y el número de partículas derivando con respecto a $\b\m$ del logaritmo de la función partición.

\underline{Nota}: Como estamos considerando bosones \textit{sin masa} no importa si $T$ es grande o pequeño porque el \textit{potencial químico $\m$ es cero}. Es decir, no hay costo energético de crear o introducir una partícula al sistema. Ahora, la relación de dispersión para un fotón viene dada por
\begin{equation}
  \ep(p)=pc=h\n 
\end{equation}
Entonces la energía promedio es
\begin{equation}
  j(p)=\ep (p)=pc
\end{equation}
Dado que el fotón es un bosón e gauge (mediador de fuerza) tiene spin $1$ pero solo tiene degeneración de spin igual a $2$ y no $3$ debido a que el fotón no tiene masa. Reemplazando en \eqref{16.1},
\begin{align}
  \frac{E}{V}&=2\int\frac{\dd^3p}{(2\p\hbar)^3}pc\frac{e^{-pc/\k T}}{1-e^{-pc/\k T}}\\
  &=\frac{8\p c}{(2\p\hbar)^3}\int_0^\infty p^3\dd p\frac{e^{-pc/\k T}}{1-e^{-pc/\k T}}\quad \cdot /\frac{e^{pc/\k T}}{e^{pc/\k T}}\\
  &=\frac{8\p c}{(2\p\hbar)^3}\int_0^\infty p^3\dd p \frac{1}{e^{pc/\k T}-1}
\end{align}
Haciendo $u=pc/(\k T)$, se tiene
\begin{align}
   \frac{E}{V}=\frac{8\p c}{(2\p\hbar)^3}\left(\frac{\k T}{c}\right)^4\underbrace{\int_0^\infty \frac{u^3}{e^{u}-1}}_{\p^4/15}
\end{align}
\begin{equation}
\boxed{  \frac{E}{V}=\frac{\p }{15\hbar^3c^3}(\k T)^4}
\end{equation}
Conocida como la \textit{ley de Stephen-Boltzmann} o \textit{radiación de cuerpo negro}.

Suponiendo un objeto que no deja escapar la radiación recibida, tal objeto se calienta y emite radiación solo debido a su temperatura. La mecánica estadística nos muestra que al tener un cuerpo a temperatura $T$, se crean fotones en el medio circundante (en equilibrio térmico con el cuerpo) de acuerdo a la distribución de Bose-Einstein.

Ahora, $J=N\implies j(p )=1$. Entonces
\begin{align}
  \frac{N}{V}&=2\int\frac{\dd^3p}{(2\p\hbar)^3}\frac{e^{-pc/\k T}}{1-e^{-pc/\k T}}\\
  &=\frac{8\p }{(2\p\hbar)^3}\int_0^\infty p^2\frac{e^{-pc/\k T}}{1-e^{-pc/\k T}}\dd p,\qquad u=pc/(\k T)\\
  &=\frac{8\p }{(2\p\hbar)^3}\left(\frac{\k T}{c}\right)^3\underbrace{\int_0^\infty \frac{u^2}{e^u-1}}_{2\zeta (3)}
\end{align}
\begin{equation}
  \boxed{\frac{N}{V}=\frac{2\zeta(3)}{\p^2\hbar^3}\left(\frac{\k }{c}\right)^3T^3\implies \frac{N}{V}\propto T^3}
\end{equation}

Con esto, podemos calcular la energía promedio por fotón según
\begin{equation}
  \epsilon(T)=\frac{E}{N}=\frac{E/v}{N/V}=\frac{\p^4}{30\zeta(3)}\k T\approx 2.71 \k T
\end{equation}

Recordemos que teníamos antes para una partícula en $3D$ que
\begin{equation}
  \ep (T)=3\k T 
\end{equation}
Ahora tenemos una distribución de energía (gas de fotones a temperatura $T$) habrá partículas con distinta energía.

Para calcular la densidad espectral de energía consideramos la densidad de energía calculada
\begin{equation}
  \frac{E}{V}=\frac{8\p c}{(2\p\hbar)^3}\int_0^\infty p^3\frac{\dd p}{e^{pc/\k T}-1}
\end{equation}
y además
\begin{equation}
  p=\frac{h\n }{c},\qquad \dd p=\frac{h}{c}\dd\n 
\end{equation}
Reemplazando en la densidad de energía
\begin{align}
  \frac{E}{V}&=\frac{8\p c}{(2\p\hbar)^3}\left(\frac{h}{c}\right)^4\int_0^\infty\frac{\n^3\dd\n }{e^{h\n /\k T}-1},\quad h=2\p \hbar\\
  &=\frac{8\p h}{c^3}\int_0^\infty\frac{\n^3\dd\n }{e^{h\n /\k T}-1}\\
  &=\int_0^\infty B_\n (T)\dd\n 
\end{align}
donde 
\begin{equation}
 \boxed{ B_\n (T)=\frac{8\p h\n^3}{(e^{h\n /\k T}-1)c^3}}
\end{equation}
corresponde a la \textit{función densidad de energía por unidad de frecuencia de la radiación} o distribución espectral de energía del cuerpo negro buscada.

Notemos que $B_\n (T)$ puede escribirse de la siguiente forma
\begin{equation}
  B_\n (T)=\frac{8\p T^3}{\hbar^2c^3}\mathcal{B}\left(\frac{h\n }{\k T}\right)
\end{equation}
donde 
\begin{equation}
  \mathcal{B}(x)=\frac{x^3}{e^x-1}
\end{equation}
Podemos encontrar $\n_{\rm max}(T)$ (\textit{Ley de Win}),
\begin{align}
  0&=\dv{\n}B_\n \\
  &\vdots\\
  &=(3-x)e^x-3
\end{align}
así,
\begin{equation}
\boxed{  \n_{\rm max}=\frac{2.82}{h}\k T}
\end{equation}
\end{sol}

\begin{ej}
	Ahora consideremos un gas de fermiones sin masa a temperatura $T$. Un ejemplo de fermiones sin masa son los \textit{neutrinos} (se conoce que tienen masa por las oscilaciones de neutrinos y la diferencia que hay entre los distintos tipos). \textit{En el modelo estándar no tienen masa.}
	
	Al estar presentes en el medio pueden formar un gas e neutrinos, ya que al estar en equilibrio con el \textit{baño térmico} (y no tener masa) no hay costo energético en crearos.
\end{ej}
\begin{sol}
	Siguiendo un procedimiento similar al anterior, se tiene que
	\begin{equation}
\boxed{  \frac{E}{V}=\frac{7\k^4\p^2}{\hbar^3c^4120}T^4}
\end{equation}
\begin{equation}
\boxed{  \frac{N}{V}=\frac{3}{2}\frac{\zeta(3)\k^3}{\p^2\hbar^3c^3}T^3}
\end{equation}


\end{sol}






























