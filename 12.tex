\section{Clase 12}
\subsection{Gases degenerados}
El término \textit{gases degenerados} se refiere al hecho de que hay algunos modos para los cuales la probabilidad de ocupación $f(\epsilon)$ no es mucho menor que uno. En tales casos, los bosones y fermiones se comportan de manera muy distinta.

Consideremos
\begin{equation}
  f(T-\m )=\frac{e^{-\b (\epsilon-\m )}}{1\pm e^{-\b (\epsilon-\m )}}
\end{equation}
con $f(\ep-\m )$ la función de ocupación:
\begin{align}
  +&:\quad \text{para fermiones}\\
  -&:\quad \text{para bosones}
\end{align}
Cuando $f(\ep-\m )\approx 1$ se dice que el gas está \textit{degenerado}.

\subsection{Gas degenerado de bosones y condensado de Bose-Einstein}
De \eqref{8.densidadnumerica}, la densidad numérica viene dada por
\begin{equation}
  \rho=\frac{N}{V}=(2s+1)\int \frac{\dd^Dp}{(2\p \hbar)^D}\left(\frac{e^{-(\epsilon(p)-\m)/\k T }}{1-e^{-(\epsilon(p)-\m )/\k T}}\right)
\end{equation}
Suponemos un gas no-relativista, luego la relación de dispersión está dada por
\begin{equation}
  \epsilon(p)\approx \frac{p^2}{2m}
\end{equation}
Podemos ver que para $\m \geq 0$, $f(\epsilon-\m )$ tiene una divergencia
\begin{equation}
  p\in [0,\infty),\qquad \epsilon(p)\in [0,\infty)
\end{equation}
En particular, para $\epsilon\to \m $, se tiene que $f(\epsilon-\m )\to \infty$. En este caso hay una formación de un \textit{condensado de Bose-Einstein}.

Estudiemos el caso límite $\m =0^{-}$ para encontrar la densidad requerida para la condensación de Bose:
\begin{align}
  \rho&=(2s+1)\int \frac{\dd^Dp}{(2\p \hbar)^D}\frac{e^{-\frac{p^2}{2m\k T}}}{1-e^{-\frac{p^2}{2m\k T}}}\\
  &=   (2s+1)\int \frac{\dd^Dp}{(2\p \hbar)^D}\sum_{n=0}^\infty e^{-(n+1)p^2/(2m\k T)}      \\
  &=   (2s+1)\int \frac{\dd^Dp}{(2\p \hbar)^D}\sum_{n=1}^\infty e^{-(np^2/(2m\k T)}  \\
  &=(2s+1)\int_0^\infty \frac{\Omega_D p^{D-1}\dd p}{(2\p\hbar)^D}\sum_{n=1}^\infty \left(e^{-p^2/(2m\k T)}\right)^n
\end{align}
donde $\Omega_D$ es el ángulo sólido y se usó el hecho de que
\begin{equation}
  \sum_{n=0}^\infty x^n=\frac{1}{1-x},\quad |x|<1
\end{equation}

 Haciendo el siguiente cambio de variable,
\begin{equation}
  u=\frac{p}{\sqrt{m\k T}},\qquad \dd u=\frac{\dd p}{\sqrt{m\k T}}
\end{equation}
se tiene
\begin{align}
  \rho&=(2s+1) \int_0^\infty\frac{\Omega_Du^{D-1}}{(2\p\hbar)^D}(m\k T)^{(D-1)/2}(m\k T)^{1/2}\dd u\sum_{n=1}^\infty \left(e^{-u^2/2}\right)^n\\
  &=(2s+1)\frac{(m\k T)^{D/2}}{(2\p\hbar)^D}\sum_{n=1}^\infty \int\dd^D ue^{-nu^2/2}
\end{align}

\begin{prop}
	\begin{equation}\label{12.prop}
  \sum_{n=1}^\infty \int\dd^D ue^{-nu^2/2}=\left(\int \dd^D ue^{-u^2/2}\right)\sum_{n=1}^\infty n^{-D/2}
\end{equation}
\end{prop}
\begin{prueba}
	Queremos mostrar que 
	\begin{equation}
  \int\dd^Due^{-nu^2/2}=n^{-D/2}\int\dd^Due^{-u^2/2}
\end{equation}
En efecto,
\begin{align}
  \int\dd^Due^{-nu^2/2}&=\Omega_D\int_0^\infty u^{D-1}\dd e^{-nu^2/2}
\end{align}
Haciendo
\begin{equation}
  z=\sqrt{n}u\implies u=\frac{z}{\sqrt{n}}\implies	 \dd u=\frac{\dd z}{\sqrt{n}}
\end{equation}
tenemos
\begin{align}
  \int\dd^Due^{-nu^2/2}&=\Omega_D\int_0^\infty \frac{\dd z}{\sqrt{n}}\left(\frac{z}{\sqrt{n}}\right)^{D-1}e^{-z^2/2}\\
  &=\Omega_D\int_0^\infty \dd z zn^{-D/2}e^{-z^2/2}\\
  &=n^{-D/2}\int_0^\infty \dd z^De^{-z^2/2}\\
  &=n^{-D/2}\int_0^\infty \dd u^De^{-u^2/2}
\end{align}
luego, se sigue \eqref{12.prop}.
\end{prueba}


Así, $\r$ nos queda
\begin{equation}
  \boxed{\rho=(2s+1)\frac{(m\k T)^{D/2}}{(2\p\hbar)^D}\left( \int\dd^D ue^{-u^2/2}\right)\sum_{n=1}^\infty n^{-D/2}}
\end{equation}

Para $D=1$ ó $D=2$ la sumatoria es divergente. Luego, $\rho$ puede ser arbitrariamente grande sin que llegue al valor $0$. Es decir, $\forall \r >0, \exists \m <0$. Entonces \textit{no hay} condensación de Bose-Einstein.

Para $D\geq 3$,
\begin{equation}
  \sumn1 n^{-D/2}
\end{equation}
es convergente!. En particular \footnote{Recordar que la función zeta de Riemann se define como
\begin{equation}
 \z (s)= \sumn1 \frac{1}{n^s }.\end{equation}
}
\begin{equation}
  \boxed{\sumn1 n^{-D/2}=\z \left(\frac{D}{2}\right)}
\end{equation}

Entonces para $D=3$,
\begin{align}
  \r(\m =0,T)=(2s+1)\frac{(m\k T)^{3/2}}{(2\p\hbar)^3}\underbrace{\left(4\p \int_0^\infty u^2e^{-u^2/2}\dd u\right)}_{(2\p)^{3/2}}\z\left(\frac{3}{2}\right)
\end{align}
donde $u^2$ es la Jacobiano. Luego, usando la propiedad
\begin{prop}
	\begin{equation}
  \int_0^\infty u^2e^{-u^2/2}\dd u=\sqrt{\frac{\p }{2	}}
\end{equation}

\end{prop}

nos queda
\begin{equation}
  \boxed{\r_c(T)=(2s+1)\frac{(m\k T)}{(2\p)^{3/2}\hbar^3}\z \left(\frac{3}{2}\right)}
\end{equation}

Notar que $\r_c(T)$ es la densidad de condensación en función de la temperatura, es decir, es la densidad a la cual el potencial químico $\m$ se hace cero. En general, $\m$ es negativo \textit{y se interpreta como el costo de energía por partícula de poner una nueva en el sistema.}

Cuando a $T$ fija, $\r>\r_c(T)$, el sistema se encuentra en dos fases:
\begin{enumerate}
	\item Una fase a densidad $\r_c(T)$ descrita por la función de ocupación de Bose-Einstein a $\m=0$ ($f(T,\m=0)$).
	\item Una fase a densidad $(\r-\r_c(T))$ tal que todas las partículas se encuentran en el estado fundamental. A esta fase se le llama \textit{condensado de Bose-Einstein.}
\end{enumerate}

\subsection{Gas degenerado de Fermi}
En el caso de los fermiones, la función de ocupación viene dada por
\begin{equation}
  \boxed{f(\ep-\m,\b )=\frac{e^{-\b(\epsilon-\m  )}}{1+e^{-\b(\epsilon-\m)}}}
\end{equation}
Notar que para $\b\to \infty \,(T\to 0)$, se tiene
\begin{equation}\label{12.Theta}
  f(\epsilon-\m ,\b )=\Th (\m-\epsilon )= \left\{ \begin{array}{lcc} 1,& \epsilon<\m \\ \\ 
  \frac{1}{2},&\epsilon=\m \\ \\  0,&\epsilon>\m  \end{array}\right.
\end{equation}
De acá se desprende que
\begin{equation}
  \frac{e^{-\b(\epsilon-\m  )}}{1+e^{-\b(\epsilon-\m)}}= \left\{\begin{array}{lcc}
  	\lim_{x\to\infty}\frac{x}{1+x}=1\\ \\\
  	\lim_{x\to 0}\frac{0}{1+0}=0
  \end{array}   \right.
\end{equation}
Esto se entiende del hecho de que como máximo puede haber un fermión en cada estado.

Considere la densidad numérica a $T=0$,
\begin{equation}
  \rho=\frac{N}{V}=(2s+1)\int \frac{\dd^Dp}{(2\p \hbar)^D}f(\epsilon-\m ,T)
\end{equation}
a $T=0$, de \eqref{12.Theta} se tiene
\begin{equation}
  \rho=\frac{N}{V}=(2s+1)\int \frac{\dd^Dp}{(2\p \hbar)^D} \Th (\m-\epsilon)
\end{equation}
Entonces de la función de Heaviside la cual fija los límites de la integral,
\begin{equation}
  \r =(2s+1)\int_0^{p (\epsilon=\m )}\frac{\Omega_D p^{D-1}\dd p}{(2\p\hbar)^D}
\end{equation}
pero para una partícula no-relativista se tiene
\begin{equation}
  \epsilon=\frac{p^2}{2m},\implies  p(\epsilon=\m )=\sqrt{2m\m }
\end{equation}
luego tenemos
\begin{equation}
  \r =(2s+1)\frac{\Omega_D}{(2\p\hbar)^D}\frac{1}{D}(\sqrt{2m\m })^D
\end{equation}
Finalmente
\begin{equation}
 \boxed{ \r=(2s+1)\frac{\Omega_D}{D(2\p\hbar)^D}(2m\epsilon_f)^{D/2}}
\end{equation}
donde $\m$ a $T=0$ se le llama \textit{energía de Fermi} $\epsilon_f$.

El potencial químico para un gas de Fermi es positivo. Representa una \textit{ganancia energética por el solo hecho de que una partícula ocupe cierto estado.}

Notar que a $T=0$, todos los estados con energía $\epsilon<\epsilon_f$ están ocupados (con una partícula en cada estado).

Para $T\geq 0$ se tiene que $f(\epsilon-\m )$ es una función que interpola entre $0$ para $(\m-\epsilon)\leq 0$ y $1$ para $(\m-\epsilon)\geq 0$.



\begin{ej}
	Use 
	\begin{equation}
  P=-\left(\pdv{E}{V}\right)_{S,N}
\end{equation}
para encontrar $P(N)$ la presión degenerada de Fermi como función del número de fermiones en el gas (use $s=1/2)$. Para $m=m_n$ transforme a $P(M)$. ¿Cuál es la presión en el interior de una estrella de neutrones de masa $M$? (Desprecie factores relativistas)

\underline{Hints:} \begin{itemize}
	\item Use $\r (\epsilon_f)$.
	\item Encontrar $\ev{E}/V$ como función de $\epsilon_f$.
	\item Despejar $\epsilon_f(\r )$ y reemplazar en $\ev{E}/V$ para encontrar $\ev{E}(V,N)$ a $T=0$.
	\item Usar 
	\begin{equation}\label{12.P}
  P=-\left(\pdv{E}{V}\right)_{S,N}
\end{equation}
\end{itemize}
\end{ej}

\begin{sol}
	Tenemos
	\begin{equation}
  f(\ep-\m )=\frac{e^{-\b(\ep-\m )}}{1+e^{-\b(\ep_\m )}}
\end{equation}
\begin{equation}\label{12.rho}
\r(\ep )=(2s+1)\frac{\Omega_D}{D(2\p\hbar)^D}(2m\epsilon_f)^{D/2}
\end{equation}
\begin{equation}
  \frac{\ev{E}}{V}=(2s+1)\int_{\mathbb{R}^D}\frac{\dd^Dp}{(2\p\hbar)^D}\epsilon(p)f(\ep-\m ,\b )
\end{equation}
Utilizando la relación de dispersión no-relativista y que a $T=0, f(\ep-\m )=\Th(\ep-\m )$ en la última expresión, tenemos
\begin{align}
  \frac{\ev{E}}{V}&=\frac{2}{(2\p\hbar)^3}\int\dd^3p\left(\frac{p^2}{2m}\right)\Th (\ep-\m )\\
  &=\frac{2}{(2\p\hbar)^3}\int\Omega_3p^2\dd p\left(\frac{p^2}{2m}\right)\Th (\ep-\m )\\
  &=\frac{2}{(2\p\hbar)^3}\int_{p=0}^{p=\sqrt{2m\m }}4\p \frac{p^4}{2m}\dd p
\end{align}
Recordando que el potencial químico $\m$ a $T=0$ es la energía de Fermi, se tiene
\begin{align}
   \frac{\ev{E}}{V}&=\frac{4\p }{(2\p\hbar)^3m}\frac{1}{5}(2m\ep_f )^{5/2}\\
   &=\frac{1}{2\p^2\hbar^3 m}\frac{1}{5}(2m\ep_f )^{5/2}\label{12.E/V}
\end{align}
Ahora, de \eqref{12.rho} tenemos
\begin{equation}
  \r(\ep_f)=\frac{2}{3}\frac{4\p }{(2\p \hbar)^3}(2m\epsilon_f)^{3/2}=\frac{8\p }{3(2\p \hbar)^3}(2m\epsilon_f)^{3/2}
\end{equation}
usando que $\r=N/V$, podemos despejar la energía de Fermi en términos de $N$ y $V$,
\begin{equation}
  \ep_f=\left(\frac{N}{V}\frac{3(2\p\hbar)^3}{8\p (2m)^{3/2}}\right)^{2/3}
\end{equation}
Reemplazando en \eqref{12.E/V} y multiplicando por $V$ a ambos lados para obtener la energía promedio,
\begin{equation}
  \ev{E}=\underbrace{\frac{(2m)^{5/2}}{10\p\hbar^3m}\left(\frac{3(2\p\hbar)^3}{8\p (2m)^{3/2}}\right)^{5/3}}_{C}\frac{N^{5/3}}{V^{2/3}}
\end{equation}
\begin{equation}
  \ev{E}=C\frac{N^{5/3}}{V^{2/3}}\implies \pdv{E}{V}=-\frac{2}{3}C\left(\frac{N}{V}\right)^{5/3}
\end{equation}
Así de \eqref{12.P}, nos queda
\begin{equation}
  \boxed{P(N)=-\frac{2}{3}C\left(\frac{N}{V}\right)^{5/3}}
\end{equation}
Podemos encontrar $P$ en función de la masa total $M$ ya que si conocemos el número de partículas y la masa de una partícula $m$, 
\begin{equation}
  M=Nm,\qquad  N=\frac{M}{m}
\end{equation}
resultando
\begin{equation}
 \boxed{ P(M)=-\frac{2}{3}C\left(\frac{M}{mV}\right)^{5/3}}
\end{equation}
\end{sol}





































