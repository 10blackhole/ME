\section{Clase 17}
\subsection{Tiempo de evaporación de un agujero negro}
\begin{ej}
	Considere un agujero negro de masa $M$, cuya temperatura de Hawking es
	\begin{equation}
  T_H=\frac{\hbar c^3}{8\p G\k_p M}
\end{equation}
suponiendo que producto de la exitación del vacío circundante en el agujero negro, se producen solamente fotones. Calcule el tiempo de evaporación.
\end{ej}

\begin{sol}
	Sabemos que la densidad de energía es
	\begin{equation}
  \frac{E}{V}=aT^4
\end{equation}
Sea $F$ el flujo de energía por unidad de área por unidad de tiempo, el cual es un flujo radiativo que proviene del horizonte de eventos,
\begin{equation}
  F=\sigma T^4
\end{equation}
donde $\sigma$ es la costante de flujo de Boltzmann. Además
\begin{equation}
  r_s=\frac{2MG}{c^2},\quad A=4\p r_s^2,\quad \sigma=\frac{ca}{4}
\end{equation}
Necesitamos encontrar una ecuación diferencial para la masa en función del tiempo ya que la evaporación ocurre cuando $M=0$. Si usamos el flujo de energía, el cual tiene unidades de $E/At$ podemos encontrar una funcón que depende solo de $M$ y $t$, entonces multiplciamos por el área para eliminar la dependencia en las unidades,
\begin{equation}
  4\p r_s^2F=4\p r^2T^4
\end{equation}
Sabemos además que $E=mc^2$, entonces dividimos por $c^2$,
%TODO terminar de escribir 





\end{sol}