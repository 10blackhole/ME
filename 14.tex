\section{Clase 14}
\begin{ej}
	Use el teorema de equipartición generalizado para calcular la energía cinética de un gas de N partículas libres de masa $m$ a temperatura $T$, considerando el Hamiltoniano para cada partícula dado por
	\begin{equation}
  H=H(\vec{p})=\sqrt{p^2c^2+m^2c^4}
\end{equation}
Considere la energía cinética restando la energía cinética de la masa en reposo $mc^2$. Verifique los casos ultra-relativista y no-relativista.
\end{ej}

\begin{sol}
	El teorema de equipartición generalizado para un $H=H(\t )$ con $\t=(p,q)$ es
	\begin{equation}
  \ev{\t\pdv{H}{t}}=\k T
\end{equation}
tenemos
\begin{equation}
  H=H(\vec{p})=\sqrt{p^2c^2+m^2c^4}
\end{equation}
\begin{align}
  \pdv{H}{p_i}&=\frac{1}{2}\frac{1}{\sqrt{p^2c^2+m^2c^4}}2p_ic^2\\
  &=\frac{p_ic^2}{\sqrt{p^2c^2+m^2c^4}}
\end{align}
Reemplazamos
\begin{align}
  \ev{p_i\frac{p_ic^2}{\sqrt{p^2c^2+m^2c^4}}}&=\k T\\
  \ev{\frac{p_i^2c^2}{\sqrt{p^2c^2+m^2c^4}}}&=\k T
\end{align}
Estamos considerando (hasta ahora) el momentum en una sola dirección. Como estamos en $3$-dimensiones
\begin{equation}
  \sum_{i=1}^3\frac{p_i^2c^2}{\sqrt{p^2c^2+m^2c^4}}=3\k T
\end{equation}
Como sumo en $i$
\begin{equation}
  \frac{c^2\sum_{i=1}^3p_i^2}{\sqrt{p^2c^2+m^2c^4}}=3\k T
\end{equation}
notemos que la suma $p_x^2+p_y^2+p_z^2=\vec{p}^2$. Entonces
\begin{equation}
  \boxed{\frac{c^2\vec{p}^2}{\sqrt{p^2c^2+m^2c^4}}=3\k T}
\end{equation}
Ahora para encontrar la energía cinética (como el $H$ que nos da el enunciado)
\begin{equation}
  c^2p^2=\a 
\end{equation}
\begin{align}
  \implies \a&=3\k T\sqrt{\a +m^2c^4}\quad /()^2\\
  \a^2&=9\k^2T^2(\a +m^2c^4)\\
  \a^2-9\k^2T^2\a &=9\k^2T^2m^2c^4\\
   \a^2-9\k^2T^2\a -9\k^2T^2m^2c^4&=0
\end{align}
Usamos 
\begin{equation}
  x=\frac{-b\pm\sqrt{b^2-4ac}}{2a}
\end{equation}
Nos queda
\begin{equation}
  \a=\frac{9\k^2T^2\pm\sqrt{(9\k^2T^2)^2+4\times 9\k^2T^2m^2c^4}}{2}
\end{equation}
Elegimos la raíz positiva, ya que $\vec{p}^2c^2\geq 0$,
\begin{align}
  \a&=\frac{9\k T^2}{2}+\frac{\sqrt{(9\k^2T^2)^2+36\k^2T^2m^2c^4}}{2}\\
  &=\frac{9\k T^2}{2}+\left(\frac{(9\k^2T^2)^2}{4}+\frac{36\k^2T^2m^2c^4}{4}\right)^{1/2}\\
  &=\frac{9\k T^2}{2}+\left(\left(\frac{9\k^2T^2}{2}\right)^2+9\k^2T^2m^2c^4\right)^{1/2}
\end{align}
Entonces, obtenemos
\begin{equation}
 \boxed{ \a=\vec{p}^2c^2=\frac{9\k T^2}{2}+\left(\left(\frac{9\k^2T^2}{2}\right)^2+9\k^2T^2m^2c^4\right)^{1/2}}
\end{equation}
Construimos la energía cinética
\begin{equation}\label{14.1}
 \boxed{ E_c=\sqrt{\underbrace{\frac{9\k T^2}{2}+\left(\left(\frac{9\k^2T^2}{2}\right)^2+9\k^2T^2m^2c^4\right)^{1/2}}_{\vec{p}^2c^2}+m^2c^4}-mc^2}
\end{equation}
donde $mc^2$ s la energía de la partícula en reposo.

Ahora, vamos a verificar las cosas que se piden


\underline{\textbf{Caso 1} Ultra-relativista}: $\k T\gg mc^2$
\begin{align}
  E_c&=\sqrt{\frac{9\k T^2}{2}+\left(\left(\frac{9\k^2T^2}{2}\right)^2+9\k^2T^2m^2c^4\right)^{1/2}+m^2c^4}-mc^2\\
  &\approx \left[\frac{9\k^2T^2}{2}+\frac{9\k^2T^2}{2}\right]^{1/2}\\
  &=3\k T
\end{align}

\underline{\textbf{Caso 2} no-relativista}: $\k T\ll mc^2$. Notamos que no podemos simplemente despreciar los términos con $\k T$ ya que nos quedaría que
\begin{equation}
  E_c=mc^2-mc^2=0
\end{equation}
Usaremos $x\ll 1$,
\begin{equation}\label{14.star}
  (1+x)^n\approx 1+nx
\end{equation}
Primero, reescribimos
\begin{align}
  E_c&=\frac{3\k T}{\sqrt{2}}\left(\left[1+\left(1+\frac{m^2c^49\k^2T^2}{\left(\frac{9\k^2T^2}{2}\right)^2}\right)^{1/2}+\frac{m^2c^4}{\frac{9\k^2T^2}{2}}\right]-\frac{mc^2}{\frac{3\k T}{\sqrt{2}}}\right)
\end{align}
Usando \eqref{14.star},
\begin{align}
   E_c&=\frac{3\k T}{\sqrt{2}}\left(\left[1+1+\frac{1}{2}\frac{4m^2c^49\k^2T^2}{(9\k^2T^2)^2}+\frac{2m^2c^4}{9\k^2T^2	}\right]^{1/2}-\frac{\sqrt{2}mc^2}{3\k T}\right)\\
   &=\frac{3\k T}{\sqrt{2}}\left(\sqrt{2+\frac{4m^2c^4}{9\k^2T^2}}-\frac{\sqrt{2}mc^2}{3\k T}\right)\\
   &=\frac{3\k T}{\sqrt{2}}\sqrt{2}\left(\left(1+\frac{2m^2c^4}{9\k^2T^2}\right)^{1/2}-\frac{\sqrt{2}mc^2}{3\k T}\right)
\end{align}
usando nuevamente \eqref{14.star},
\begin{align}
   E_c&=3\k T\left(\left(1+\frac{m^2c^4}{9\k^2T^2}\right)-\frac{mc^2}{3\k T}\right)\\
   &=3\k T+\frac{m^2c^4}{3\k T}-mc^2
\end{align}
\begin{equation}
	\implies E_c=3\k T(1+m^2c^4)-mc^2
\end{equation}
Pero este NO es el resultado correcto!!!

Notemos que \eqref{14.1} se puede reescribir como,
\begin{equation*}
  E_c(T,N,m)=\frac{3N\k T}{\sqrt{2}}\left[\left(1+\left(1+\frac{4}{9}\left(\frac{mc^2}{\k T}\right)^2\right)^{1/2}+\frac{2}{9}\left(\frac{mc^2}{\k T}\right)^2\right)^{1/2}-\frac{\sqrt{2}}{3}\left(\frac{mc^2}{\k T}\right)\right]
\end{equation*}
la cual corresponde a la energía cinética del gas considerando $N$ partículas a temperatura $T$. Sea
\begin{equation}
  \lambda=\frac{mc^2}{\k T}=\left\{\begin{array}{cc}
  	\lambda \ll 1,& \text{límite ultra-relativista}\\\\
  	\lambda\gg 1,&\text{límite no-relativista}
  \end{array}\right.
\end{equation}
Entonces
\begin{equation}
  E_c(T,N,m)=\frac{3N\k T}{2}\underbrace{\left[\sqrt{2}\left(1+\left(1+\frac{4}{9}\lambda^2\right)^{1/2}+\frac{2}{9}\lambda^2\right)^{1/2}-\frac{2}{3}\lambda\right]}_{F(\lambda)}
\end{equation}
Así, tenemos una $F(\lambda)$ que interpola entre $2$ a $\lambda\ll 1$ y $1$ a $\lambda\gg 1$.

Analicemos para el caso de $\lambda\gg 1$,
\begin{align}
  1+\frac{4}{9}\lambda^2&\sim \frac{4}{9}\lambda^2\\
  1+\frac{2}{9}\lambda^2&\sim \frac{2}{9}\lambda^2
\end{align}
Entonces
\begin{align}
  F(\lambda\g 1)&\approx \sqrt{2}\sqrt{\frac{2}{9}\lambda^2+\frac{2}{3}\lambda^2}-\frac{2}{3}\lambda \\
  &\approx \sqrt{2}\frac{\sqrt{2}}{3}\lambda\left(1+\frac{3}{\lambda}\right)^{1/2}-\frac{2}{3}\lambda\\
  &\approx \frac{2}{3}\lambda\left(1+\frac{3}{2\lambda}\right)-\frac{2}{3}\lambda\\
  &\approx 1
\end{align}
el cual corresponde al límite no-relativsta. Luego,
\begin{equation}
\boxed{  E_c=\frac{3N\k T}{2}}
\end{equation}











\end{sol}

\subsection{Paradoja de Gibbs y partículas idénticas}
Para $N$ partículas \textit{idénticas} (que no interactúan entre sí)
\begin{equation}
 \boxed{ \zc (N,T)=\frac{Z(T)^N}{N!}}
\end{equation}
donde $Z(T)$ es la función partición de una partícula.

El factor $N!$ se necesita para no contar múltiples veces las configuraciones donde las $N$ partículas ocupan estados diferentes, sin que importe que partícula ocupa tal estado.

\begin{ej}
	Considere $N=2$,
	\begin{equation}
  \zc=\frac{1}{2}\left(\sumi e^{-\b \ep_i}\right)^2=\frac{1}{2}\sum_{i,j}e^{-\b\ep_i-\b\ep_j}
\end{equation}
donde $\ep_i$ representa la energía del estado ocupado por la partícula $1$ y $\ep_j$ representa la de la partícula $2$. Pero si ambas partículas son indistinguibles, no hay difrencia entre la configuración dada por
\begin{center}
\begin{tabular}{|c|c|}
\hline
 Energía&Partícula\\\hline
 $\ep_i$&$1$\\\hline
 $\ep_j$&$2$ \\\hline 
\end{tabular}
\end{center}
o por
\begin{center}
\begin{tabular}{|c|c|}
\hline
 Energía&Partícula\\\hline
 $\ep_j$&$1$\\\hline
 $\ep_i$&$2$ \\\hline 
\end{tabular}
\end{center}
Luego, el factor $1/2$ corrige el efecto de sumar dos veces la misma configuración equivalente.

El argumento se generaliza de forma trivial a $N$ partículas.

Veamos el efecto que esto tiene sobre la función partición y la entropía de un gas no-relativista de partículas idénticas a temperatura $T$.


Se tiene, en el ensamble canónico ($(N,T,V)$ constantes) que
\begin{align}
  S&=\ln\zc+\b\ep \\
  &=\ln\left(\frac{Z(T)^N}{N!}\right)+\b\ep\\
  &=N\ln(Z(T))-\ln(N!)+\b\ep\\
 &= N\ln(Z(T))-N\ln(N)+N+\b\ep 
\end{align}
donde se usó la aproximación de Stirling.
Entonces, la función partición para una partícula libre no-relativista
\begin{equation}
 \boxed{ Z(T)=\frac{V}{(2\p\hbar)^3}\int\dd^3\vec{p}e^{-\vec{p}^2/2m\k T}=V\left(\frac{m\k T}{2\p\hbar }\right)^{3/2}}
\end{equation}
La función partición canónica para un gas ideal no-relativista
\begin{equation}
  \boxed{\zc(N,T)=\frac{V^N}{N!}\left(\frac{m\k T}{2\p\hbar^2}\right)^{3N/2}}
\end{equation}
\begin{equation}
  \implies \boxed{\frac{S}{N}=\ln\left[\frac{V}{N}\left(\frac{m\k T}{2\p\hbar^2}\right)^{3/2}\right]+\frac{5}{2}}
\end{equation}
Vemos que el efecto del factor $1/N!$, proveniente del hecho que sn partículas idénticas, es hacer que la entropía de una partícula depende de $V/N$, es decir, del volumen ocupado \textit{por partícula}.
\end{ej}































