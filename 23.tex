\section{Clase 23}
Teniamos de la clase anterior que la energía or molécula para un gas ideal diatómico, dependia de la temperatura del gas y está dada por:
\begin{equation}
  \ev{\frac{E}{N}}=\left\{\begin{array}{ccc}
  	\dfrac{3}{2}\k T,& \k T>\hbar^2/2I,& \text{$T$ menor del nivel fundamental para que la molécula rote}\\\\
  	\dfrac{5}{2}\k T,& \hbar^2/2I\leq \k T<\hbar\omega_{\rm vib},& \text{$T$ mayor para que rote pero no lo suficiente para que vibre}\\\\
  	\dfrac{7}{2}\k T,& \k T\geq \hbar\omega_{\rm vib},& \text{$T$ mayor para que el nivel fundamental de vibración.}
  \end{array}\right.
\end{equation}

%TODO img

Para los gases típicos a temperatura ambiente se tiene\footnote{Por ejemplo, Oxígeno molecular y el Hidrógeno molecular.}
\begin{equation}
  \frac{\hbar^2}{2I}\ll \k T_{\rm ambiente}\ll \hbar\omega_{\rm vib}
\end{equation}

Notar que $I$ es el momento de inercia el cual esta dado por $mr_0^2/4$ para una molécula diatómica de masa $m$.

Para un gas diatómico de átomos de la misma masa y separación interatómica de equilibrio $r_0$, el momento de inercia $I$ se calcula de acuerdo a lo siguiente
\begin{align}
  I&=\frac{m}{2}\left(\frac{r_0}{2}\right)^2+\frac{m}{2}\left(\frac{r_0}{2}\right)2\\
  I&=\frac{mr_0^2}{4}
\end{align}
donde 
\begin{equation}
  I=\sumi m_ir_i^2
\end{equation}
Finalmente $\omega_{\rm vib}$ es la frecuencia natual de vibración de la molécula ($\k=m\omega^2$).

Encontremos el calor específico a volúmen constante $C_V$. Por definición
\begin{equation}
  C_V=\frac{1}{N\k }T\eval{\dv{S}{T}}_{N,V}
\end{equation}
y
\begin{equation}
  \dd E=T\dd S-P\dd V+\m\dd N
\end{equation}
luego, a $N$ y $V$ constante
\begin{equation}
  \eval{\dd E}_{N,V}=\eval{T\dd S}_{N,V}
\end{equation}
por lo tanto, tengo que
\begin{equation}
  \eval{T\pdv{S}{T}}_{N,V}=\eval{\pdv{E}{T}}_{N,V}
\end{equation}
de donde se obtiene lo siguiente
\begin{equation}
  \frac{1}{N\k }\eval{T\pdv{S}{T}}_{N,V}=\frac{1}{N\k }\eval{\pdv{E}{T}}_{N,V}
\end{equation}
a $N$ constante
\begin{equation}
  \frac{1}{N\k }\eval{\pdv{E}{T}}_{N,V}=\frac{1}{\k }\eval{\pdv{(E/N)}{T}}_{N,V}
\end{equation}
pero tenemos el valor de $E/N$, luego
\begin{equation}
  C_V=\left\{\begin{array}{ccc}
  	\dfrac{3}{2},&& \k T>\hbar^2/2I\\\\
  	\dfrac{5}{2},& &\hbar^2/2I\leq \k T<\hbar\omega_{\rm vib}\\\\
  	\dfrac{7}{2},&& \k T\geq \hbar\omega_{\rm vib}
  \end{array}\right.
\end{equation}
Ahora encontramos e calor específico a presión constante $C_P$
\begin{equation}
  C_P=\frac{1}{N\k }\eval{T\pdv{S}{T}}_{N,P}
\end{equation}
de la primera ley
\begin{equation}
  \dd E=T\dd S-P\dd V+\m\cancelto{0}{\dd N}
\end{equation}
Ahora, usando la ecuación de estado $PV=N\k T$ tenemos que
\begin{align}
  \eval{\dd (PV)}_{N,P}&=\eval{\dd(N\k T)}_{N,P}\\
  \eval{P\dd V}_{N,P}&=\eval{N\k \dd T}_{N,P}
\end{align}
Luego,
\begin{equation}
  \eval{-P\dd V}_{N,P}=\eval{-N\k \dd T}_{N,P}
\end{equation}
y por lo tanto
\begin{align}
  \eval{\dd E}_{N,P}+\eval{P\dd V}_{N,P}&=\eval{T\dd S}_{N,P}\\
  \eval{\dd E}_{N,P}+\eval{N\k \dd T}_{N,P}&=\eval{T\dd S}_{N,P}
\end{align}
Pero, $E=C_VN\k T$ y por tanto
\begin{equation}
  \eval{\dd E}_{N}=\eval{C_VN\k \dd T}_N
\end{equation}
En particular
\begin{equation}
  \eval{\dd E}_{N,P}=\eval{C_VN\k \dd T}_{N,P}
\end{equation}
y entonces reemplazamos
\begin{align}
  \eval{C_VN\k\dd T}_{N,P}+\eval{N\k\d T}_{N,P}&=\eval{T\dd S}_{N,P}\\
  C_P&=\frac{1}{N\k }\eval{T\pdv{S}{T}}_{N,P}=C_V+1
\end{align}
Entonces,
\begin{equation}
  C_P=\left\{\begin{array}{ccc}
  	\dfrac{5}{2},&& \k T>\hbar^2/2I\\\\
  	\dfrac{7}{2},& &\hbar^2/2I\leq \k T<\hbar\omega_{\rm vib}\\\\
  	\dfrac{9}{2},&& \k T\geq \hbar\omega_{\rm vib}
  \end{array}\right.
\end{equation}

\subsection{Ejemplos de procesos termodinámicos que ivolucran al gas diatómico}
\begin{ej}
	Considere una expansión isoentrópica de un gas diluiod de moleculas $O_2$ a temperatuta cercana a ambiente, donde el volúmen cambia de $V_a$ a $V_b$.
	\begin{enumerate}
		\item Encuentre la temperatura $T_b$ en términos de la temperatura inicial $T_a$
		\item Encuentre la presión $P_b$ en términos de la presión original $P_a$ y los volumenes $V_b$ y $V_a$.
	\end{enumerate}
	
	\underline{Nota}: Gas diludo equivale a la ecuación del gas ideal. Esto porque en general, $A_1=1$ en la expansión del virial, para cualquier ecuación de estado.
\end{ej}

\begin{sol}
	Tenemos 
	\begin{align}
  PV&=N\k T\\
  E&=\frac{5}{2}N\k T
\end{align}
Ahora, de la primera ley
\begin{equation}
  \dd E=T\dd S-P\dd V+\m \dd N
\end{equation}
pero $\dd S=0$ por que es una expansión adiabática y $\dd N=0$ porque estamos a $N$ constante. Así,
\begin{equation}
  -P\dd V=\dd E
\end{equation}
Haciendo algo de cálculo directo se tiene
\begin{align}
  \ln\left(\frac{V_a}{V_b}\right)&=-\frac{5}{2}\ln\left(\frac{T_b}{T_a}\right)\\
  \frac{T_a}{T_b}&=\left(\frac{V_b}{V_a}\right)^{-2/5}\\
  \Aboxed{T_b&=T_a\left(\frac{V_b}{V_a}\right)^{-2/5}}
\end{align}




\end{sol}












