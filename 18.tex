\section{Clase 18}
\subsection{Ecuación de Van der Waals y transiciones de fase}
La ecuación de estado de Van der Waals es
\begin{equation}
\boxed{  P=\frac{\r\k T}{1-\r/\r_s}-a\r^2}
\end{equation}
donde $\r=N/V$ es la densidad de número, $\r_s=1/v$ es el volumen excluido o volumen unitario y $a$ es el parámetro de atracción. Además $0\leq\r\leq\r_s$ de manera que $0\leq P\leq \infty$.

Notemos que el término $-a\r^2$ es atractivo, en el sentido de que disminuye la presión del fluido. Por otro lado, para $\r\approx\r_s$ tiene un comportamiento repulsivo ya que la presión del fluido disminuye.

Para $\r$ pequeño, $P\to \r\k T$, es decir, a densidades bajas, el fluido de Van der Waals se comporta como un gas ideal.

La ecuación de estado corresponde al comportamiento dado por la siguiente aproximación de campo medio
\begin{equation}
  \phi(\vec{r})=\left\{\begin{array}{cc}
  	+\infty,&\quad |\vec{r}|\leq r_0 \\\\
  	u,&\quad |\vec{r}|> r_0 
  \end{array}\right.
\end{equation}
Este potencial corresponde a uno de \textit{esfera sólida} a cierto radio mínimo.

Consideremos las definiciones
\begin{equation}\label{18.volexc}
  \frac{4}{3}\p r_0^3=vN
\end{equation}
donde $vN$ es el volumen excluido total, y
\begin{equation}
  u=-\frac{aN}{V}
\end{equation}
es la magnitud del potencial atractivo el cual es proporcional a la densidad de partículas.

Ahora podemos construir la función de partición canónica del sistema (fluido de VdW).
\begin{align}
  \zc &=\frac{1}{N!}\left((2s+1)\int\frac{\dd^3p\dd^3r}{(2\p\hbar)^3}e^{-(\epsilon(p)+\f (r))\k T}\right)^N\\
  &=\frac{1}{N!}(\xi_p)^N\left(\int_0^\infty 4\p r^2e^{-\f(r)\/\k T}\dd r\right)^N
\end{align}
donde
\begin{equation}
  \xi_p=(2s+1)\int\frac{\dd^3p}{(2\p\hbar)^3}e^{-\epsilon(p)/k T}
\end{equation}
Notando que $e^{-\infty}=0$ y que ademas $e^{-\f(r)/\k T}=e^{-u/\k T}=$ const. para $r>r_0$, tenemos que
\begin{equation}
  \int_0^\infty 4\p r^2e^{-\f(r)/\k T}\dd r=(V-vN)e^{-u/\k T}
\end{equation}
En efecto,
\begin{align}
  \int_0^\infty 4\p r^2e^{-\f(r)/\k T}\dd r&=\cancelto{0}{\int_0^{r_0} 4\p r^2e^{-\f(r)/\k T}\dd r}+\int_{r_0}^\infty 4\p r^2e^{-\f(r)/\k T}\dd r\\
  &=e^{-u/\k T}\int_{r_0}^\infty 4\p r^2\dd r\\
  &=e^{-u/\k T}\left[\int_{0}^\infty 4\p r^2\dd r- \int_{0}^{r_0} 4\p r^2\dd r\right]\\
  &=e^{-u/\k T}(V-vN)
\end{align}
donde usamos la definición del volumen excluido \eqref{18.volexc}. Así
\begin{equation}\label{18.z}
 \boxed{ \zc=\frac{1}{N!}(\xi_p)^N(V-vN)^Ne^{-u/\k T}}
\end{equation}
Finalmente para derivar la ecuación de estado, consideramos
\begin{equation}
  F=E-TS,\qquad F=-\k T\ln\zc 
\end{equation}
\begin{equation}
  \dd F=-S\dd T-P\dd V+\m \dd N
\end{equation}
luego,
\begin{equation}
  P=-\left(\pdv{F}{V}\right)_{T,N}=\k T=\left(\pdv{(\ln\zc)}{V}\right)_{T,N}
\end{equation}
de \eqref{18.z} y haciendo algo de álgebra se tiene
\begin{equation}
  P=\k T\left(\frac{-N^2a}{\k T V^2}+\frac{N}{V-vN}\right)
\end{equation}
y usando que
\begin{equation}
  \r=\frac{N}{V},\qquad \r_s=\frac{1}{v}
\end{equation}
obtenemos la ecuación de estado de Van der Waals,
\begin{equation}
\boxed{  P=\frac{\r\k T}{1-\r/\r_s}-a\r^2}
\end{equation}

La \textit{ecuación de estado} de un sistema relaciona las variables de dicho sistema entre sí.



































