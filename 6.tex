\section{Clase 6}
%De la clase anterior, tenemos la siguiente tabla resúmen
%
%\begin{center}
%\begin{tabular}{|c|c|c|c|}
%\hline
%  Ensamble & Potencial & Definición &Variación del potencial \\
%  \hline
%  Microcanónico &Entropía $S(E,V,Q)$& $S$ & $\dd S=\frac{1}{T}\dd E+\frac{P}{T}\dd V-\frac{\m }{T}\dd Q$ \\\hline
%  Canónico&Energía libre de Hemholtz $F(T,V,Q)$ &$F=E-TS=\m Q-PV$\\\hline
%  Gran Canónico&Potencial gran canónico $(\Ogc)$&$(T,V,\m )$\\\hline
%  Isobárico-Isotérmico&Energía libre de Gibbs $(G)$&$(T,P,Q)$\\\hline
%  Joule-Thompson&Entalpía $(H)$&$(S,P,Q)$\\\hline
%\end{tabular}
%\end{center}

\subsection{Fuerzas a temperatura finita}
El trabajo por hecho por una fuerza al recorrer una distancia $\dd\vec{x}$, está dado por
\begin{equation}
  W=-\int_{\vec{x}_1}^{\vec{x}_2}\vec{f}\cdot\dd\vec{x}
\end{equation}
Esto es una contribución a la energía del sistema.

Supongamos un movimiento en $\vec{x}$. Luego incluyendo la contribución de $W$ a la energía del sistema, la cual funcionalmente es análoga al termino $-P\dd V$, tenemos
\begin{equation}\label{6.star}
  \dd E=T\dd S-P\dd V+\m\dd Q-f\dd x
\end{equation}
donde $f=|\vec{f}_x|$ es la magnitud de la fuerza.

Ahora, suponamos que el sistema está a $T$ constante. Consideremos $F(T,V,Q)$ y la relacionamos con $f$:
\begin{equation}
  F=E-TS
\end{equation}
\begin{equation}
  \dd F=\dd E-S\dd T-T\dd S
\end{equation}
Considerando \eqref{6.star} se tiene
\begin{equation}
  \dd F=-S\dd T-P\dd V+\m \dd Q-f\dd x
\end{equation}
\begin{equation}
  \implies f\dd x=-\dd F-S\dd t-P\dd V+\m\dd Q
\end{equation}
si consideramos $(T,V,Q)$ fijos, se tiene
\begin{equation}
  f\dd x=-\dd F
\end{equation}
luego,
\begin{equation}
 \boxed{ f=-\eval{\left(\pdv{F}{x}\right)}_{T,V,Q}}
\end{equation}
es decir, la energía libre de Hemholtz $F=E-TS$ juega el rol de potencial.

Notemos que $F$ puede obtenerse directamente de la función partición canónica,
\begin{equation}
  \boxed{F=-\k T\ln (\zc)}
\end{equation}

\begin{ej}
	Un ejemplo estándar consiste en derivar la ley de Hooke para una cadena de $N$ eslabones a temperatura $T$.
\end{ej}

\subsection{Relaciones de Maxwell}
Las relaciones de Maxwell consisten en un conjunto de equivalencias entre derivadas de variables termodinámicas.

Todas las relaciones de Maxwell pueden ser derivadas desde la relación fundamental de la termodinámica,
\begin{equation}
  \dd S=\b\dd E+(\b P)\dd V-(\b\m )\dd Q
\end{equation}
de donde podemos identificar directamente,
\begin{align}
  \b=\eval{\pdv{S}{E}}_{V,Q}\\
  \b P=\eval{\pdv{S}{V}}_{E,Q}\\
  \b\m=-\eval{\pdv{S}{Q}}_{V,E}
\end{align}

\begin{ej}
	Considere
	\begin{equation}
  \dd S=\frac{1}{T}\dd E+\frac{P}{T}\dd V-\frac{\m }{T}\dd Q
\end{equation}
Notamos que para $F(x,y)$ continua y $2$ veces diferenciable, se tiene
\begin{equation}
  \pdv{F}{x}{y}=\pdv{F}{y}{x}
\end{equation}
es decir, sus derivadas conmutan. Luego, por ejemplo
\begin{equation}
  \pdv{S}{E}{V}=\pdv{S}{V}{E}
\end{equation}
\begin{equation}
	\left(\pdv{E}\left(\pdv{S}{V}\right)_{E,Q}\right)_{V,Q}=\left(\pdv{V}\left(\pdv{S}{E}\right)_{V,Q}\right)_{E,Q}
\end{equation}
\begin{equation}
  \implies \boxed{\pdv{E}\left(\frac{P}{T}\right)_{V,Q}=\pdv{V}\left(\frac{1}{T}\right)_{E,Q}}
\end{equation}
\end{ej}
Notar que cada potencial termodinámico depende de tres variables y por lo tanto se pueden escoger tres diferentes relaciones entre dobles derivadas. Luego como son $5$ potenciales termodinámicos diferentes (para cada uno de los $5$ ensambles), tenemos $15$ relaciones diferentes entre sus dobles derivadas. \textbf{Estas son $15$ relaciones de Maxwell.}

\begin{ej}
	Consideremos $G=G(T,P,Q)$, así
	\begin{equation}
  \dd G=-S\dd T+V\dd P+\m \dd Q
\end{equation}
Ahora
\begin{equation}
  \pdv{G}{T}{P}=\pdv{G}{P}{T}
\end{equation}
\begin{equation}
  \left(\pdv{T}\left(\pdv{G}{P}\right)_{T,Q}\right)_{P,Q}=\left(\pdv{P}\left(\pdv{G}{T}\right)_{P,Q}\right)_{T,Q}
\end{equation}
\begin{equation}
  \implies \boxed{\eval{\pdv{T}{V}}_{P,Q}=-\eval{\pdv{S}{P}}_{T,Q}}
\end{equation}
\end{ej}

Consideremos otro ejemplo.
\begin{ej}
	Muestre que 
	\begin{equation}
  \left(\pdv{E}{Q}\right)_{S,P}=\m -P\left(\pdv{\m}{P}\right)_{S,Q}
\end{equation}
Vemos que $V$ no aparece. Luego, despejamos la relación de $V$ de la segunda ley
\begin{equation}
  \dd E=T\dd S-P\dd V+\m\dd Q
\end{equation}
\begin{equation}\label{6.2star}
  \implies \dd V=\frac{T}{P}\dd S+\frac{\m }{P}\dd Q-\frac{1}{P}\dd E
\end{equation}
De aquí podemos considerar, sumando $E/P$ para eliminar la dependencia de $\dd E$,
\begin{equation}\label{6.mio}
  \dd\left(V+\frac{E}{P}\right)=\dd V+\frac{\dd E}{\dd P}+E\dd\left(\frac{1}{P}\right)
\end{equation}
Pasando el último término de \eqref{6.2star} hacia el lado derecho y reemplazando \eqref{6.mio},
\begin{equation}
  \dd\left(V+\frac{E}{P}\right)=\frac{T}{P}\dd S+\frac{\m }{P}\dd Q-E\dd\left(\frac{1}{P}\right)
\end{equation}
Entonces 
\begin{equation}
  \pdv{(V+E/P)}{(1/P)}{Q}=\pdv{(V+E/P)}{Q}{(1/P)}
\end{equation}
Notemos que si y sólo si $(1/P)$ es constante,
\begin{equation}
  \left(\pdv{(1/P)}\left(\pdv{\m }{P}\right)\right)_{S,Q}=\left(\pdv{Q}E\right)_{S,P}
\end{equation}
Usando que
\begin{equation}
  \pdv{(1/P)}=-P^2\pdv{P}
\end{equation}
tenemos
\begin{equation}
  \left(\pdv{E}{Q}\right)_{S,P}=-P^2\left(\pdv{P}\left(\pdv{\m }{P}\right)\right)_{S,Q}
\end{equation}
\begin{equation}
   \left(\pdv{E}{Q}\right)_{S,P}=-P^2\left(-\frac{1}{P^2}\m +\frac{1}{P}\left(\frac{\m }{P}\right)_{S,Q}\right)
\end{equation}
\begin{equation}
  \boxed{ \left(\pdv{E}{Q}\right)_{S,P}=\m -P\left(\pdv{\m }{P}\right)_{S,Q}}
\end{equation}
\end{ej}

\subsection{Fluctuaciones}
De la función de partición es posible obtener momentos de orden superior para las variables termodinámicas \footnote{Los momentos de orden superior se pueden entender como derivadas de orden superior de la función partición con respecto a los multiplicadores de Lagrange.} . Por ejemplo, tenemos
\begin{equation}
  \zc=\sumi e^{-\b E_i}
\end{equation}
y por definición la energía promedio viene dada por
\begin{equation}
  \ev{E}=\sumi E_iP_i
\end{equation}
y
\begin{equation}
  P_i=\frac{e^{-\b E_i}}{\zc}
\end{equation}
por lo tanto, de \eqref{E=partial-beta},
\begin{equation}
  \ev{E}=\frac{\sumi E_i e^{-\b E_i}}{\zc}=-\frac{1}{\zc}\pdv{\zc}{(1/\k T)}
\end{equation}
Entonces
\begin{equation}
  \boxed{\ev{E}=-\pdv{(1/\k T)}\ln (\zc)=\k T^2\pdv{T}\ln (\zc)}
\end{equation}
Además,
\begin{equation}
  \ev{E^2}=\sumi E_i^2 P_i=\frac{\sumi E_i^2 e^{-Ei/\k T}}{\zc}
\end{equation}
%Vemos que
%\begin{equation}
%  \boxed{\pdv[2]{\b }\ln (\zc)=(E-\ev{E})^2}
%\end{equation}
%notar que $(E-\bar{E})^2$ es la varianza de $E$. 
Notar que dado que
\begin{equation}
  \beta=\frac{1}{\k T}
\end{equation}
en términos de $\b$ tenemos
\begin{equation}
  \zc=\sumi e^{-\b E_i}
\end{equation}
luego
\begin{align}
  \pdv{\b}\ln (\zc)&=-\frac{1}{\zc}\sumi E_i\eb\\
  &=-\ev{E}
\end{align}
es decir,
\begin{equation}
  \boxed{\pdv{\b}\ln (\zc)=-\ev{E}}
\end{equation}

Calculemos ahora la segunda derivada con respecto a $\b$,
\begin{align}
  \pdv[2]{\b}\ln(\zc)&=\pdv{\b}\left(-\frac{1}{\zc}\sumi E_i\eb\right)\\
  &=\left(\frac{1}{\zc^2}\pdv{\b}\sumi E_i\eb\right)+\frac{1}{\zc}\sumi E_i^2\eb\\
  &=\left(-\frac{1}{\zc^2}E_j\sum_j e^{-\b E_j}\right)\left(\sumi E_i\eb\right)+\frac{1}{\zc}\sumi E_i^2\eb\\
  &=\left(-\frac{1}{\zc}E_j\sum_j e^{-\b E_j}\right)\left(\frac{1}{\zc}\sumi E_i\eb\right)+\frac{1}{\zc}\sumi E_i^2\eb\\
  &=-\ev{E}^2+\ev{E^2}
\end{align}
así,
\begin{equation}\label{6.d2blnz}
 \boxed{ \pdv[2]{\b}\ln(\zc)=\ev{E^2}-\ev{E}^2}
\end{equation}

\begin{prop}
Sea $\ev{E}$ el valor promedio de una cantidad, entonces la varianza de $E$ se puede escribir de dos maneras equivalentes
	\begin{equation}
  \text{Var}({E})=\ev{(E-\ev{E})^2}=\ev{E^2}-\ev{E}^2
\end{equation}
\end{prop}
En efecto,
\begin{align}
  \ev{(E-\ev{E})^2}&=\sumi P_i\left(E_i-\sumj P_j E_j\right)^2\\
  &=\sumi P_i\left(E_i^2-2E_i\sumj P_jE_j+\left(\sumj P_jE_j\right)^2\right)\\
  &=\sumi P_iE_i^2-2\left(\sumi P_iE_i\right)\left(\sumj P_jE_j\right)+\underbrace{\sumi P_i}_{1}\left(\sumj P_jE_j\right)^2\\
  &=\ev{E^2}-2\ev{E}^2+\ev{E}^2\\
  &=\ev{E^2}-\ev{E}^2
\end{align}
Luego, \eqref{6.d2blnz} se puede escribir como
\begin{equation}
  \boxed{\pdv[2]{\b}\ln(\zc)=\ev{(E-\ev{E})^2}\equiv \text{Var}(E)}
\end{equation}

Además, podemos reescribir esto en términos de derivadas de $T$, usando
\begin{equation}
  \b=\frac{1}{\k T}\quad \implies \qquad \pdv{\b}=-\k T^2\pdv{T}
\end{equation}
luego,
\begin{align}
  \pdv{\b}\left(\pdv{\b}\ln\zc\right)&=\pdv{\b}\left(-\k T^2\pdv{T}\ln\zc\right)\\
  &=\k^2 T^2\pdv{T}\left(T^2\pdv{T}\ln\zc\right)\\
  &=\k^2T^2\left(2T\pdv{T}\ln\zc+T^2\pdv[2]{T}\ln\zc\right)
\end{align}
finalmente, se tiene
\begin{equation}
  \boxed{\text{Var}(E)=2\k^2T^3\pdv{T}\ln\zc+\k^2T^4\pdv[2]{T}\ln\zc}
\end{equation}








%\begin{align}
%  \pdv[2]{\b}\ln(\zc)&=-\pdv{\b}\left(\frac{1}{\zc}\sumi E_i\eb \right)\\
%  &=\left(\frac{1}{\zc^2}\sumi E_i\eb\right)\left(\sumi E_i\eb\right)+\frac{1}{\zc}\sumi E_i^2\eb\\
%  &=-\ev{E}^2-\ev{E^2}\\
%  &=\ev{E^2}-\ev{E}^2
%\end{align}
%es decir,
%\begin{equation}
%   \pdv[2]{\b}\ln(\zc)=\ev{E^2}-\ev{E}^2
%\end{equation}
%Luego,
%%TODO checkear
%
%\begin{equation}
%  \ev{(E-\ev{E})^2}=\sumi P_i (E_i-\ev{E})^2=
%\end{equation}
















